% Created 2023-02-11 Sat 13:52
% Intended LaTeX compiler: pdflatex
\documentclass[11pt]{article}
\usepackage[utf8]{inputenc}
\usepackage[T1]{fontenc}
\usepackage{graphicx}
\usepackage{longtable}
\usepackage{wrapfig}
\usepackage{rotating}
\usepackage[normalem]{ulem}
\usepackage{amsmath}
\usepackage{amssymb}
\usepackage{capt-of}
\usepackage{hyperref}
\author{Yusheng Zhao}
\date{\today}
\title{Assignment 1}
\hypersetup{
 pdfauthor={Yusheng Zhao},
 pdftitle={Assignment 1},
 pdfkeywords={},
 pdfsubject={},
 pdfcreator={Emacs 28.2 (Org mode 9.6)}, 
 pdflang={English}}
\begin{document}

\maketitle

\section{Problem 1}
\label{sec:org4cd41c3}
We will be using the following formula to calculate FLOPS of my CPU

\[FLOPS = cpu\_speed \times num\_cores \times avx\_factor \times fma\_factor \]

Now, we will collect information of above parameters from our CPU. However,
MacOS does not support \texttt{lsmem} and \texttt{lscpu}. The best alternative I could find on
the internet is to execute the following scirpt

\begin{verbatim}
sysctl -a | grep machdep.cpu | fold -w60
\end{verbatim}

\begin{verbatim}
machdep.cpu.cores_per_package: 10
machdep.cpu.core_count: 10
machdep.cpu.logical_per_package: 10
machdep.cpu.thread_count: 10
machdep.cpu.brand_string: Apple M1 Max
machdep.cpu.features: FPU VME DE PSE TSC MSR PAE MCE CX8 API
C SEP MTRR PGE MCA CMOV PAT PSE36 CLFSH DS ACPI MMX FXSR SSE
 SSE2 SS HTT TM PBE SSE3 PCLMULQDQ DTSE64 MON DSCPL VMX EST 
TM2 SSSE3 CX16 TPR PDCM SSE4.1 SSE4.2 AES SEGLIM64
machdep.cpu.feature_bits: 151121000215084031
machdep.cpu.family: 6
\end{verbatim}

However, I could not find any wanted parameters here. Therefore, I just \texttt{ssh}
into another linux machine and got its info instead. Here is the output for
\texttt{lscpu} on that machine.
\begin{verbatim}
Architecture:                    x86_64
CPU op-mode(s):                  32-bit, 64-bit
Address sizes:                   46 bits physical, 57 bits v
irtual
Byte Order:                      Little Endian
CPU(s):                          8
On-line CPU(s) list:             0-7
Vendor ID:                       GenuineIntel
Model name:                      Intel Xeon Processor (Icela
ke)
CPU family:                      6
Model:                           134
Thread(s) per core:              1
Core(s) per socket:              1
Socket(s):                       8
Stepping:                        0
BogoMIPS:                        5985.93
Flags:                           fpu vme de pse tsc msr pae 
mce cx8 apic sep mtrr pge mca cmov pat pse36 clflush mmx fxs
r sse sse2 ss syscall nx pdpe1gb rdtscp lm constant_tsc rep_
good nopl xtopology cpuid tsc_known_freq pni pclmulqdq vmx s
sse3 fma cx16 pcid sse4_1 sse4_2 x2apic movbe popcnt tsc_dea
dline_timer aes xsave avx f16c rdrand hypervisor lahf_lm abm
 3dnowprefetch cpuid_fault invpcid_single ssbd ibrs ibpb sti
bp ibrs_enhanced tpr_shadow vnmi flexpriority ept vpid ept_a
d fsgsbase tsc_adjust bmi1 avx2 smep bmi2 erms invpcid avx51
2f avx512dq rdseed adx smap avx512ifma clflushopt clwb avx51
2cd sha_ni avx512bw avx512vl xsaveopt xsavec xgetbv1 xsaves 
wbnoinvd arat avx512vbmi umip pku ospke avx512_vbmi2 gfni va
es vpclmulqdq avx512_vnni avx512_bitalg avx512_vpopcntdq la5
7 rdpid fsrm md_clear arch_capabilities
Virtualization:                  VT-x
Hypervisor vendor:               KVM
Virtualization type:             full
L1d cache:                       256 KiB (8 instances)
L1i cache:                       256 KiB (8 instances)
L2 cache:                        32 MiB (8 instances)
L3 cache:                        128 MiB (8 instances)
NUMA node(s):                    1
NUMA node0 CPU(s):               0-7
Vulnerability Itlb multihit:     Not affected
Vulnerability L1tf:              Not affected
Vulnerability Mds:               Not affected
Vulnerability Meltdown:          Not affected
Vulnerability Mmio stale data:   Vulnerable: Clear CPU buffe
rs attempted, no microcode; SMT Host state unknown
Vulnerability Retbleed:          Not affected
Vulnerability Spec store bypass: Mitigation; Speculative Sto
re Bypass disabled via prctl and seccomp
Vulnerability Spectre v1:        Mitigation; usercopy/swapgs
 barriers and __user pointer sanitization
Vulnerability Spectre v2:        Mitigation; Enhanced IBRS, 
IBPB conditional, RSB filling, PBRSB-eIBRS Not affected
Vulnerability Srbds:             Not affected
Vulnerability Tsx async abort:   Mitigation; TSX disabled
\end{verbatim}

I notice, I have \(8\) cores and they support \texttt{avx512f} and \texttt{fma}. According to
this \href{https://stackoverflow.com/a/59146690}{answer}, we adjust \texttt{avx\_fct} and \texttt{fma\_fct} used in calculation accordingly.

I notice the \texttt{lscpu} reports \texttt{BogoMIPS} instead of \texttt{cpu speed}. So I did \texttt{cat
/proc/cpuinfo | grep cpu} to find out the cpu clock speed is in fact \texttt{2992.969
MHz}.

The FLOPS is calculated by the following code.
\begin{verbatim}
begin
    cpu_speed = 2.992969; # GHz
    num_cores = 8;
    avx_fct = 512 / 64; # 2 floats
    fma_fct = 2;
    println("The computing power of my CPU is \
            $(cpu_speed * num_cores * avx_fct * fma_fct) GFLOPS")
end
\end{verbatim}

\begin{verbatim}
The computing power of my CPU is 383.100032 GFLOPS
\end{verbatim}


Thas is extremely fast. However, we also need to consider data I/O bottleneck.
Meaning, we need to take data from \textbf{DRAM} to \textbf{SRAM} on the CPU. According to the
lecture, this is about \texttt{100 GB/s}. Each GFLOP uses roughly \texttt{32 *2 GB} of data on
\texttt{x64} machine.

\begin{verbatim}
begin
    io_speed = 100 #GB/s
    gflop_data = 32 * 2 #GB , we operate on two floats
    println("data I/O bottleneck caps the machine speed to \
                $(io_speed/gflop_data) GFLOPS")
 end
\end{verbatim}

\begin{verbatim}
data I/O bottleneck caps the machine speed to 1.5625 GFLOPS
\end{verbatim}


Lastly, if the program requires multiple data I/O, we will then hit the latency
problem. The output of \texttt{lsmem} shows that we have \texttt{64GB} of ram, the OS will
take away approximately \texttt{1GB}. So we could use \texttt{63GB} to do useful things. Every
\texttt{63GB} of data, we need extra \texttt{50ns} of time to let the machine know we need new
data!
\begin{verbatim}
RANGE                                 SIZE  STATE REMOVABLE 
BLOCK
0x0000000000000000-0x00000000bfffffff   3G online       yes 
  0-2
0x0000000100000000-0x000000103fffffff  61G online       yes 
 4-64

Memory block size:         1G
Total online memory:      64G
Total offline memory:      0B
\end{verbatim}

\begin{verbatim}
begin
    mem_size = 63; #GB
    gflop_data = 32 * 2 ; #GB
    latency = 50 /(10^9); # s
    println("Latency issue caps our machine speed at \
         $(mem_size/gflop_data/(1+latency)) GFLOPS")
end
\end{verbatim}

\begin{verbatim}
Latency issue caps our machine speed at 0.9843749507812526 GFLOPS
\end{verbatim}



\textbf{In conclusion}, for a short amount of time, our machine could do work at
\texttt{383.100032 GFLOPS}. Then it will hit a data I/O bottleneck and degrade to
\texttt{1.52625 GFLOPS}. Eventually, it will hit the latency bottleneck and degrade to
less than \texttt{0.9843749507812526 GFLOPS}.

\section{Problem 2}
\label{sec:org1c6248b}
Let us assume information is propogating at the speed of light. And latency time
is the time it took starting from CPU issues a data request til the time data
starts to arrive at the cache.

\begin{verbatim}
begin
    c = 29979245800 # cm/s
    dist = 10 # cm
    cpu_freq =  2.992969 # GHz
    println("The latency is $(dist*2/c) seconds")
    println("The CPU clock time is $(1/(cpu_freq * 10^9)) seconds")
end
\end{verbatim}

\begin{verbatim}
The latency is 6.671281903963041e-10 seconds
The CPU clock time is 3.341163907811942e-10 seconds
\end{verbatim}


The minimum latency time is \textbf{roughly twice} as the cpu clock. But we were told
in class that it's much longer. Probably due to the need to actually find and
retrieve those data in DRAM.

\section{Problem 3}
\label{sec:orgce55d6f}

\section{Problem 4}
\label{sec:orgb4acbd6}
\end{document}